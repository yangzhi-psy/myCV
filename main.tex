%%%%%%%%%%%%%%%%%%%%%%%%%%%%%%%%%%%%%%%%%
% "ModernCV" CV and Cover Letter
% LaTeX Template
% Version 1.11 (19/6/14)
%
% This template has been downloaded from:
% http://www.LaTeXTemplates.com
%
% Original author:
% Xavier Danaux (xdanaux@gmail.com)
%
% License:
% CC BY-NC-SA 3.0 (http://creativecommons.org/licenses/by-nc-sa/3.0/)
%
% Important note:
% This template requires the moderncv.cls and .sty files to be in the same 
% directory as this .tex file. These files provide the resume style and themes 
% used for structuring the document.
%
%%%%%%%%%%%%%%%%%%%%%%%%%%%%%%%%%%%%%%%%%

%----------------------------------------------------------------------------------------
%	PACKAGES AND OTHER DOCUMENT CONFIGURATIONS
%----------------------------------------------------------------------------------------

\documentclass[12pt,a4paper,sans]{moderncv} % Font sizes: 10, 11, or 12; paper sizes: a4paper, letterpaper, a5paper, legalpaper, executivepaper or landscape; font families: sans or roman

\moderncvstyle{casual} % CV theme - options include: 'casual' (default), 'classic', 'oldstyle' and 'banking'
\moderncvcolor{green} % CV color - options include: 'blue' (default), 'orange', 'green', 'red', 'purple', 'grey' and 'black'

\usepackage{lipsum} % Used for inserting dummy 'Lorem ipsum' text into the template

\usepackage[scale=0.75]{geometry} % Reduce document margins
%\setlength{\hintscolumnwidth}{3cm} % Uncomment to change the width of the dates column
%\setlength{\makecvtitlenamewidth}{10cm} % For the 'classic' style, uncomment to adjust the width of the space allocated to your name

%----------------------------------------------------------------------------------------
%	NAME AND CONTACT INFORMATION SECTION
%----------------------------------------------------------------------------------------

\firstname{Zhi} % Your first name
\familyname{Yang Ph.D} % Your last name
% All information in this block is optional, comment out any lines you don't need
\title{Professor of Cognitive Neuroscience}
\address{Addr}{600 South Wanping Road, Shanghai, 200030}
\mobile{(+86)18611710840}
\email{yangz@smhc.org.cn}
%\photo[70pt][0.4pt]{pictures/picture} % The first bracket is the picture height, the second is the thickness of the frame around the picture (0pt for no frame)
\quote{Laboratory of Psychological Health and Imaging\\Shanghai Mental Health Center\\Shanghai Jiao Tong University, Medical School}

%----------------------------------------------------------------------------------------

\begin{document}
\makecvtitle % Print the CV title

\section{Research Interests}
\cvitem{}{\textbf{Neuroimaging data-mining methodologies}}
\cvitem{}{\textbf{Brain development and child/adolescent psychiatry}}

%----------------------------------------------------------------------------------------
%	EDUCATION SECTION
%----------------------------------------------------------------------------------------

\section{Education}

\cventry{2003--2008}{Ph.D., Cognitive Neuroscience}{Institute of Psychology, Chinese Academy of Sciences}{Beijing, China}{\textit{Cognitive Neuroscience}}{Thesis: Reproducibility-based Independent Component Analysis for fMRI data}  % Arguments not required can be left empty
\cventry{2005--2007}{Visiting Student}{Emory University}{Atlanta, GA, USA}{\textit{Department of Biomedical Engineering}}{Advisor: Dr. Xiaoping P. Hu}
\cventry{1999--2003}{B.S., Biomedical Engineering}{Tsinghua University}{Beijing, China}{\textit{Department of Electrical Engineering}}{}

%------------------------------------------------

%\subsection{PROJECT}

%\cventry{2015}{Foot Step Power Generation by Piezoelectric Material}{Live Project.}{}{}{This is basically a Live Project. The main theme of the project is to generate power by applying pressure on piezoelectric material with the help of non-renewable resources like foot step and other.}

%----------------------------------------------------------------------------------------
%	Professional History
%----------------------------------------------------------------------------------------

\section{Professional History}
\cventry{2018--Present}{Professor}{Institute of Psychological and Behavioral Sciences, Shanghai Jiao Tong University}{Shanghai, China}{}{}
\cventry{2018--Present}{Executive Director}{Neuroimaging Data Center, Shanghai Mental Health Center}{Shanghai, China}{}{}
\cventry{2017--Present}{P.I. of Laboratory of Psychological Health and Imaging}{Shanghai Mental Health Center}{Shanghai, China}{}{}
\cventry{2015-2018}{Professor}{University of Chinese Academy of Sciences}{Beijing, China}{}{}
\cventry{2012-2017}{Associate Professor}{Institute of Psychology, Chinese Academy of Sciences}{Beijing, China}{\textit{Laboratory of Cognition and Development}}{}
\cventry{2012-2014}{Senior Research Fellow}{National Institutes of Health}{Bethesda, MD, USA}{\textit{Section on Functional Imaging Methods}}{}
\cventry{2008-2012}{Assistant Professor}{Institute of Psychology, Chinese Academy of Sciences}{Beijing, China}{\textit{Laboratory of Cognition and Development}}{}

%----------------------------------------------------------------------------------------
%	Funds
%----------------------------------------------------------------------------------------

\section{Research Funds}
\cventry{2020-2023}{Regular Project (81971682), National Natural Science Foundation of China}{Individualized diagnosis of mental disorders based on cross-disease neuroiamging database}{Role: PI}{CNY 660,000}{}

\cventry{2016-2019}{Regular Project (81571756), National Natural Science Foundation of China}{Detecting subtypes of schizophrenia using neuroimaging under natural stimulus}{Role: PI}{CNY 672,000}{}

\cventry{2013-2016}{Renewed Regular Project (81270023), National Natural Science Foundation of China}{The specificity of the neuroimaging biomarkers for mental disorders}{Role: PI}{CNY 700,000}{}

\cventry{2010-2012}{Young Investigator Project(30900366), National Natural Science Foundation of China}{Data-driven neuroimaging marker research on classification of mental disorders}{Role: PI}{CNY 210,000}{}

\cventry{2015-2017}{Beijing Nova Program (2015079B), Beijing Municipal Commission of Science and Techology}{Neuroimaging biomarkers for mental disorders: methodologies for the big data era}{Role: PI}{CNY 350,000}{}

\cventry{2017-2020}{Gaofeng Clinical Medicine Grant Support (20171929), Shanghai Municipal Education Commission}{Neuroimaging markers for schizophrenia}{Role: PI}{CNY 1,000,000}{}

\cventry{2018-2021}{Hundred-talent Fund (2018BR17), Shanghai Municipal Commission of Health}{Brain development abnormality in social anxiety disorder}{Role: PI}{CNY 900,000}{}

\cventry{2015-2019}{Youth Innovation Fundation, Chinese Academy of Sciences}{}{Role: PI}{CNY 400,000}{}

\cventry{2014-2019}{Outstanding Young Investigator Award, Institute of Psychology, Chinese Academy of Sciences}{Data-ming framework for brain network informed subject community detection}{Role: PI}{CNY 400,000}{}

\cventry{2007-2011}{The National Basic Research Program of China (The 973 Program)(2007CB512300)}{Mechanism of genetic-environment interactions in depression and schizophrenia}{Role: Participant}{Undercontract: CNY 100,000}{}

%----------------------------------------------------------------------------------------
%	Publication
%----------------------------------------------------------------------------------------

\section{Peer-Reviewed Publications (Selected)}
\cvitem{}{Deng Z, Wu J, Gao J, Hu Y, Zhang Y, Wang Y, Dong H, \textbf{\underline{Yang Z*}}, Zuo, X-N (2019). Segregated precuneus network and default mode network in aturalistic imaging. Brain Structure and Function. In Press. DOI: 10.1007/s00429-019-01953-2.}

\cvitem{}{Zhang Y, Xu Li, Hu Y, Wu J, Li C, Wang J*, \textbf{\underline{Yang Z*}} (2019). Functional connectivity between sensory-motor sub-networks reflects the duration of untreated psychosis and predicts treatment outcome of first-episode drug-naive schizophrenia. Biological Psychiatry: Cognitive Neuroscience and Neuroimaging. 4(8):697-705.}

\cvitem{}{Jiang L, Cao X, Jiang J, Li T, Wang J, {\underline{\textbf{Yang Z*}}}, Li C* (2019). Atrophy of hippocampal subfield CA2/3 in healthy elderly men is related to educational attainment. Neurobiology of Aging. 80:21-28.}

\cvitem{}{Hu Y, Du W, Zhang Y, Li N, Han Y*, \textbf{\underline{Yang Z*}} (2019). Loss of parietal memory network integrity in Alzheimer's disease. Frontiers in Aging Neuroscience. 11:67.}

\cvitem{}{Wang J, Hu Y, Li H, Ge L, Li J, Cheng L, \textbf{\underline{Yang Z*}}, Zuo XN, Xu Y* (2018). Connecting openness and the resting-state brain network: A discover-validate approach. Frontiers in Neuroscience. 12:762.}

\cvitem{}{\textbf{\underline{Yang Z}}, Zuo XN*, McMahon KL, Craddock RC, Kelly C, de Zubicaray GI, Hickie I, Bandettini PA, Castellanos FX, Milham MP*, Wright MJ (2016). Genetic and environmental contributions to functional connectivity architecture of the human brain. Cerebral Cortex. 26:2341-2352.}

\cvitem{}{\textbf{\underline{Yang Z*}}, Qiu J, Wang P, Liu R, Zuo X* (2016). Brain structure-function associations identified in large-scale neuroimaging data. Brain Structure \& Function. 221:4459-4474.}

\cvitem{}{Hu Y, Wang J, Li C, Wang Y-S, \textbf{\underline{Yang Z*}}, Zuo X-N (2016). Segregation between the parietal memory network and the default mode network: effects of spatial smoothing and model order in ICA. Science Bulletin. 61 (24):1844-1854.}

\cvitem{}{\textbf{\underline{Yang Z*}}, Huang Z, Gonzalez-Castillo J, Dai R, Northoff G, Bandettini P (2014). Using fMRI to decode true thoughts independent of intention to conceal. NeuroImage 99, 80-92.}

\cvitem{}{\textbf{\underline{Yang Z*}}, Chang C, Xu T, Jiang L, Handwerker D, Castellanos F, Milham M, Bandettini P, Zuo X* (2014). Connectivity trajectory across lifespan differentiates the precuneus from the default network. NeuroImage 89, 45-56.}

\cvitem{}{\textbf{\underline{Yang Z*}}, Zuo X, Wang P, Li Z, Laconte S, Bandettini PA, Hu X (2012). Generalized RAICAR: Discover homogeneous subject (sub)groups by reproducibility of their intrinsic connectivity networks. NeuroImage 63, 403-414.}

\cvitem{}{\textbf{\underline{Yang Z}}, Xu Y*, Xu T, Hoy C, Handwerker D, Chen G, Northoff G, Zuo X*, Bandettini P (2014). Brain network informed subject community detection in early-onset schizophrenia. Scientific Reports 4, 5549.}

\cvitem{}{\textbf{\underline{Yang Z}}, LaConte S, Weng X, Hu X* (2008). Ranking and averaging independent component analysis by reproducibility (RAICAR).  Human Brain Mapping 29, 711-725.}

\cvitem{}{Xu T, \textbf{\underline{Yang Z (co-first author)}}, Jiang L, Xing XX, Zuo XN* (2015). A connectome computation system for discovery science of brain. Science Bulletin 60, 86-95.}

\cvitem{}{\textbf{\underline{Yang Z*}}, Fang F, Weng X (2012). Recent developments in multivariate pattern analysis for functional MRI.  Neuroscience Bulletin 28, 399-408.}

\cvitem{}{\textbf{\underline{Yang Z*}}, Wu P, Weng X, Bandettini P (2014). Cerebellum engages in automation of verb-generation skill.  Journal of Integrative Neuroscience 13, 1-17.}

\cvitem{}{Liu C, Li F, Li S, Wang Y, Tie C, Wu H, Zhou Z, Zhang D, Dong J, \textbf{\underline{Yang Z*}}, Wang C* (2012). Abnormal baseline brain activity in bipolar depression: A resting-state functional magnetic resonance imaging study. Psychiatry Research: Neuroimaging 203, 175-179}

\cvitem{}{Tang L, Liu C, Jing B, Ma X, Li H, Zhang Y, Li F, Wang Y, \textbf{\underline{Yang Z*}}, Wang C* (2014). Voxel-based morphometry study of the insular cortex in bipolar depression. Psychiatry Research: Neuroimaging 224, 89-95.}

\cvitem{}{Liu C, Ma X*, Wu X, Zhang Y, Zhou F, Li F, Tie C, Dong J, Wang Y, \textbf{\underline{Yang Z*}}, Wang C (2013). Regional homogeneity of resting-state brain abnormalities in bipolar and unipolar depression. Progress in Neuro-Psychopharmacology \& Biological Psychiatry 41, 52-59.}

\cvitem{}{\textbf{\underline{Yang Z}}, Zhao J, Jiang Y*, Li C, Wang J, Weng X*, Northoff G (2011). Altered negative Unconscious processing in major depressive disorder: An exploratory neuropsychological study.  PLoS One 6, e21881}

\cvitem{}{Li W, Cui H, Zhu Z, Kong L, Guo Q, Zhu Y, Hu Q, Zhang L, Li H, Li Q, Jiang J, Meyers J, Li J, Wang J*, \textbf{\underline{Yang Z*}}, Li C* (2016). Aberrant functional connectivity between the amygdala and the temporal pole in drug-free generalized anxiety disorder. Frontiers in Human Neuroscience 10, 549.}

\cvitem{}{Sun L, Xu H, Zhang J, Li W, Nie J, Qiu Q, Liu Y, Fang Y, \textbf{\underline{Yang Z*}}, Li X* and Xiao S* (2018). Alcohol consumption and subclinical findings on cognitive function, biochemical indexes, and cortical anatomy in cognitively normal aging Han Chinese population. Frontiers in Aging Neuroscience 10, 182.}

\cvitem{}{Xu G, Jiang Y, Ma L, \textbf{\underline{Yang Z*}}, Weng X* (2012). Similar spatial patterns of neural coding of category selectivity in FFA and VWFA under different attention conditions. Neuropsychologia 50, 862-868.}

\cvitem{}{Huang Z, Zhang X, \textbf{\underline{Yang Z*}}, Dong G, Wu J, Chan A, Weng X (2010). Verbal memory retrieval engages visual cortex in musicians.  Neuroscience 168, 179-189.}

\cvitem{}{Syed MA, \textbf{\underline{Yang Z}}, Hu XP, Deshpande G (2017). Investigating brain connectomic alterations in autism using the reproducibility of independent components derived from resting state functional mri data. Frontiers in Neuroscience 11, 459. }

%----------------------------------------------------------------------------------------
%	Publication
%----------------------------------------------------------------------------------------
\section{Teaching Experience}

\cventry{2018-2019}{Introduction to Psychology}{Undergraduate course}{30 hours}{Shanghai Jiao Tong University}{}
\cventry{2008-2018}{Statistics for Psychology Research}{Graduate course}{60 hours}{Institute of Psychology, Chinese Academy of Sciences}{}
\cventry{2012-2018}{Advanced Statistics and Machine Learning}{Graduate course}{60 hours}{Institute of Psychology, Chinese Academy of Sciences}{}
\cventry{2008-2010}{Cognitive Neuroscience}{Graduate course}{60 hours}{Institute of Psychology, Chinese Academy of Sciences}{}
\cventry{2008-2018}{Brain Development and Learning}{Graduate course}{40 hours}{Institute of Psychology, Chinese Academy of Sciences}{}
\cventry{2015-2016}{Cognitive Neuroscience}{Graduate course}{48 hours}{University of Chinese Academy of Sciences}{}

%----------------------------------------------------------------------------------------
%   Examples
%----------------------------------------------------------------------------------------
%\renewcommand{\listitemsymbol}{-~}
%\cvlistitem{Quick Learner.}
%\cvlistitem{Able to handle multiple situations at the same time.}
%\cvlistitem{Easily Adaptable.}
%\cvlistitem{Can manage time effectively}
%\cvitemwithcomment{Hindi}{Intermediate}{Can understand good}
%\cvitemwithcomment{English}{Fluent}{Conversationally fluent}

%----------------------------------------------------------------------------------------
%	Award
%----------------------------------------------------------------------------------------

\section{Awards and Honors}

\cvitem{2018}{\textbf{Hundred-talent Award}, Shanghai Municipal Commission of Health}
\cvitem{2017}{\textbf{Second Prize in Science and Technology Progress Award}, Beijing Municipal Commission of Science and Techology}
\cvitem{2017}{\textbf{Chenguang Prize for Academic Excellency}, Shanghai Jiao Tong University}
\cvitem{2016}{\textbf{Elected Member of Youth Innovation Program}, Chinese Academy of Sciences}
\cvitem{2015}{\textbf{First Prize in Science and Technology Progress Award}, The Ministry of Education of China}
\cvitem{2015}{\textbf{Beijing Nova Program}, Beijing Municipal Commission of Science and Techology}
\cvitem{2014}{\textbf{Outstanding Young Investigator Award}, Institute of Psychology, Chinese Academy of Sciences}
\cvitem{2009--2011}{\textbf{Excellent Assistant Professor Award}, Institute of Psychology, Chinese Academy of Sciences}




%----------------------------------------------------------------------------------------
%	INTERESTS SECTION
%----------------------------------------------------------------------------------------

%\section{Interests}

%\renewcommand{\listitemsymbol}{-~} % Changes the symbol used for lists

%\cvlistdoubleitem{Driving}{Photography}
%\cvlistdoubleitem{Cooking}{Basket Ball}
%\cvlistitem{Cricket}
%\section{Personal Details}
%\cvitem{DOB}{04th of Feb, 1994}
%\cvitem{Address}{D.No:59-20-7/4B,APT Backside,kakinada-533002}
%\cvitem{Languages}{Telugu,Hindi,English}
%\cvitem{Mobile No.}{+91 9492390759}
%\section{References}
%\cvitem{Name}{K.Harish}
%\cvitem{Designation}{Senior Software Engineer}
%\cvitem{Organisation}{Happiest Minds-Bangalore}
%\cvitem{Ph No}{8105543434}
%\cvitem{Mai}{harish.kmail@gmail.com}


%----------------------------------------------------------------------------------------
%	COVER LETTER
%----------------------------------------------------------------------------------------

% To remove the cover letter, comment out this entire block

%\clearpage

%\recipient{HR Department}{Corporation\\123 Pleasant Lane\\12345 City, State} % Letter recipient
%\date{\today} % Letter date
%\opening{Dear Sir or Madam,} % Opening greeting
%\closing{Sincerely yours,} % Closing phrase
%\enclosure[Attached]{curriculum vit\ae{}} % List of enclosed documents

%\makelettertitle % Print letter title

%\lipsum[1-3] % Dummy text

%\makeletterclosing % Print letter signature

%----------------------------------------------------------------------------------------

\end{document}